\documentclass[]{article}
\usepackage{float}
\usepackage{graphicx}
\usepackage{blindtext}
\usepackage{geometry}
 \geometry{
 a4paper,
 total={170mm,257mm},
 left=20mm,
 top=20mm,
 }

\title{Two Class PCA Analysis Report}
\author{Daniel Alimadadian}
\date{\today}

\begin{document}

\begin{titlepage}
    \maketitle
    \section*{Warnings}
    \section*{Data Summary}
\end{titlepage}

\section*{Principal Component Analysis (PCA) Summary}
    PCA analysis is performed using \texttt{sklearn} package version <SKLEARN_V>. \\
    Data scaling = \textbf{<SCALING>}

    \begin{figure}[h!]
        \begin{center}
            \includegraphics{<SUMMARY_FIGS>}    
        \end{center}
        \caption{PCA Summary}
    \end{figure}
    
\newpage

\section*{PC1 vs. PC2}
    Loadings selected from PC1: \textbf{<PC1_LOADINGS>}. Loadings selected from PC2: \textbf{<PC2_LOADINGS>} 

    \begin{figure}[h!]
        \begin{center}
            \includegraphics{<SUMMARY_FIGS>}
        \end{center}
        \caption{PCA scores plot. <CONTROL>: n = \textbf{<NUM_CONTROL>}, <CASE>: n = \textbf{<NUM_CASE>}}
    \end{figure}

\newpage

\section*{Ranked Loadings}

    \begin{figure}[h!]
        \begin{center}
            \includegraphics{<RANKED_LOADINGS>}
        \end{center}
        \caption{Ranked loadings plots. Top: (<LOWER_QUANTILE>, <UPPER_QUANTILE>) quantiles
        represented by dashed red lines. Bottom: significant loadings are labelled.}
    \end{figure}

% A table:
% \begin{tabular}{ c | c | c | c }
% <TABLE>
% \end{tabular}

\end{document}